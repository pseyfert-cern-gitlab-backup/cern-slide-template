\maketitle

\section{introduction}
\subsection{introduction}
%\begin{frame}[fragile]{introduction}
\begin{frame}{introduction}
\begin{itemize}
\item optimising code is sometimes considered a black art
%\begin{lstlisting}
%// branchless implementation of min/max for unsigned
%static inline unsigned min(unsigned a, unsigned b) noexcept
%{
%    const unsigned mask = -(a < b);
%    return (a \& mask) | (b \& ~mask);
%}
%static inline unsigned max(unsigned a, unsigned b) noexcept
%{
%    const unsigned mask = -(a > b);
%    return (a \& mask) | (b \& ~mask);
%}
%\end{lstlisting}
\item true to some extent -- occasionally, you need the odd dirty trick
\item but most of the time, it's down to:
\end{itemize}
\end{frame}
\begin{frame}
  \frametitle{conclusion}
  
  \begin{itemize}
      \item headache ahead
  \end{itemize}

  \vspace{.3\textheight}

  \IfFileExists{./QR2.png}{
  \footnotesize{slides (excl.\ cern logo) will appear on}
    
    \gitlablink\includegraphics[width=.2\textwidth]{./QR2.png}
}{}
\end{frame}

\appendix

\begin{frame}
  \frametitle{BACKUP}
  backup is for people who can't improvise

  (yes, that's a cheap excuse for not preparing backup slides)
\end{frame}
